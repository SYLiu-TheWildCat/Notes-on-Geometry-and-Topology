\chapter{Injective Radius in Riemannian Geometry}
In this article we are going to introduce the concept of "injective radius", which emphasizes on the maximal radius of which the exponential map $\exp_p\colon\mathbb{B}^n (\delta )\to M$ is an embedding. The main result is that
\begin{thr}\label{injective-radius-on-compact-manifolds}
    Let $(M^n ,g)$ be a compact Riemannian manifold of dimension $n$, then the injective radius $i(M)>0$. In other words, there exists a positive constant $\delta >0$ such that $\exp_p\colon\mathbb{B}^n (\delta )\to M$ is always an embedding for all $p\in M$.
\end{thr}